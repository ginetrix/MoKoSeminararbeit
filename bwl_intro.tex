\documentclass[12pt, a4paper]{article}
\usepackage[utf8]{inputenc}
\usepackage[official]{eurosym}
\usepackage{german}
\usepackage{amsmath}
\usepackage{url}


\begin{document}
\section{Einleitung}
\url{http://www.lte-anbieter.info/verfuegbarkeit/lte-verfuegbarkeit-testen.php}

\section{Grundlagen}
Um die Unternehmen \textit{China Mobile} und \textit{Deutsche Telekom} miteinander vergleichen zu können, müssen Grundbegriffe aus Betriebswirtschaftlehre und Mobilkommunikation eingeführt werden. Es folgen Erläuterungen der wichtigsten Begriffe aus beiden Teilbereichen und im Falle der betriebswirtschaftlichen Kennzahlen auch graphische Übersichten. 

\subsection{Begrifflichkeiten der Betriebswirtschaftslehre}
Bis man sich den Kennzahlen widmen kann, muss man definieren was ein Unternehmen ist und was genau ein Unternehmen wirtschaftlich macht.

\subsubsection{Unternehmensziele}

Unternehmen versuchen stets wirtschaftlich zu handeln. Dieses Handeln unterliegt dem \textit{ökonomischen Prinzip}, welches sich hauptsächlich mit zwei Aspekten befasst\cite{muller}.

\begin{enumerate}
\item Minimalprinzip: Erreichen eines Ziels mit minimalem Mitteleinsatz oder
\item Maximalprinzip: Maximierung der Zielgröße bei gegebenen Mitteln
\end{enumerate}

Zusätzlich existiert ein drittes Prinzip, das \textit{Allgemeine Extremumprinzip}, welches anstrebt ein möglichst gutes Verhältnis zwischen Aufwand und Ertrag herzustellen. Dieses ist jedoch weniger ein eigenständiges Prinzip sondern vielmehr eine Kombination bzw. ein Kompromiss zwischen  \textit{Minimal-} und \textit{Maximalprinzip}. Darüber hinaus muss das \textit{finanzielle Gleichgewicht} eingehalten werden. Dies bedeutet, dass das Unternehmen zu jeder Zeit in der Lage ist ihren Zahlungsverpflichtungen nachzugehen \cite{muller}.

Neben \textit{Erfolgszielen}, die sich hauptsächlich an Kenngrößen orientieren, gibt es auch \textit{Sachziele}, die sich davon lösen und andere Bereiche anstreben \cite{domschke}. Diese sind auch für den Bereich der Telekommunikation, mit dem sich diese Ausarbeitung befasst, relevant.

\label{unternehmensziele}
\begin{itemize}
\item \textbf{Leistungsziele:} Ziele bezüglich des Marktes bzw. des Produktes, Produktqualität Marktstellung etc.
\item \textbf{Finanzziele:} Zahlungsfähigkeit des Unternehmens \& Kapitalverfügbarkeit
\item \textbf{Führungsziele:} Ziele im Zusammenhang mit der Unternehmensstruktur, dem Führungsstil 
\item \textbf{Soziale Ziele:} gerechte Entlohnung, günstige Arbeitsbedingungen sowie Sozialleistungen
\item \textbf{Ökologische Ziele:} Ziele Bereich des Umweltschutzes
\end{itemize}

Unternehmerische Erfolge werden oftmals anhand von Kennzahlen festgemacht. Dabei existiert eine Vielzahl von verschiedensten Kennziffern, die auf die Interessen der jeweiligen Personen zugeschnitten sind. So ist beispielsweise \textit{Return on Investment} (ROI) der Quotient aus Gewinn zuzüglich der Fremdkapitalzinsen und eingesetztem Kapital \cite{domschke}. Das Fremdkapital ist hierbei dasjenige Kapital, welches dem Unternehmen von Gläubigern zur Verfügung gestellt worden ist.

\begin{equation}
\frac{\text{(Gewinn durch eingesetztes Kapital - eingesetztes Kapital)}}{\text{eingesetztes Kapital}}
\end{equation}

Die soeben genannte Kennzahl (ROI) besagt wie \glqq erfolgreich\grqq \ eine Kapitalanlage war und ist demnach für (potentielle) Investoren von großem Interesse. Bei einer Investition von $10.000\euro$ und einem beispielhaften Gewinn von $20.000\text{\euro}$ wäre der ROI $\frac{10.000\text{\euro}}{10.000\text{\euro}}$ oder $100\%$. 

Im Folgenden werden drei Kennzahlen eingeführt, die durchgehend in den Jahresberichten von China Mobile und der Deutschen Telekom gebraucht werden. Diese werden auch später in Abschnitt \ref{sec:vergleich} im direkten Vergleich einander gegenübergestellt.

\subsubsection{EBIT}

\url{http://www.vnseameo.org/ndbmai/BRF.pdf}


\subsubsection{EBITDA}

\subsubsection{Cash Flow}

\subsection{Mobilkommunikation}

\subsubsection{4G}
\subsubsection{Internet of Things}
\subsubsection{Internet Plus}

\section{China Mobile}
\subsection{Leitlinien}
\subsection{Kennzahlen}
\section{Deutsche Telekom}
\subsection{Leitlinien}
\subsection{Kennzahlen}
\section{Vergleich}
\label{sec:vergleich}
\section{Fazit}
\section{Ausblick}

\medskip

\bibliographystyle{unsrt}
\bibliography{bib}
\end{document}