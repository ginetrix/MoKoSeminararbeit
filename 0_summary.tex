% !TEX encoding = UTF-8 Unicode
\subsection*{Zusammenfassung}
\pagestyle{empty}

Mobilkommunikation hat einen erheblichen Einfluss auf die heutige Gesellschaft und damit auch auf die Industrie. Durch den stetigen technologischen Wandel stehen Unternehmen vor einer Hürde: zum einen müssen Innovationen möglichst schnell umgesetzt werden um konkurrenzfähig zu bleiben und zum anderen dürfen die Kundenbedürfnisse wie Netzverfügbarkeit oder Netzstabilität nicht darunter leiden.

In dieser Seminararbeit wird die Ausweitung und der Ausbau des Mobilfunknetzes der vierten Generation betrachtet und anhand von zwei Unternehmen gegenübergestellt. China Mobile, der weltweit größte Mobilfunkbetreiber, deckt in China bis zu 100$\%$ der Bevölkerung ab und ist an Nutzermassen unübertroffen. Die Deutsche Telekom hingegen agiert auf anderen Kontinenten und versorgt dementsprechend andere Kunden. 

Das unternehmerische Handeln im Jahre 2016 wird untersucht und aus betriebswirtschaftlicher Sicht betrachtet. Dabei wird auf Nutzerverhalten und -bedürfnisse eingegangen und es werden die finanziellen Kennzahlen erläutert und miteinander verglichen. 

Trotz der unterschiedlichen Nutzer und Lage verfolgen beide Unternehmen vergleichbare Unternehmensstrategien. Die Kennzahlen sowie die Art der Erträge und die Rückgänge in Teilbereichen der Mobilkommunikation bis hin zu Zukunftsplänen sind ebenfalls sehr ähnlich. Dies deutet darauf hin, dass Nutzer, unabhängig von Herkunftsland, ähnliche Bedürfnisse haben und dass auf unternehmerischer Ebene versucht wird diesen nachzugehen.  

